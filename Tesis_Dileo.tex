% This is the Reed College LaTeX thesis template. Most of the work
% for the document class was done by Sam Noble (SN), as well as this
% template. Later comments etc. by Ben Salzberg (BTS). Additional
% restructuring and APA support by Jess Youngberg (JY).
% Your comments and suggestions are more than welcome; please email
% them to cus@reed.edu
%
% See https://www.reed.edu/cis/help/LaTeX/index.html for help. There are a
% great bunch of help pages there, with notes on
% getting started, bibtex, etc. Go there and read it if you're not
% already familiar with LaTeX.
%
% Any line that starts with a percent symbol is a comment.
% They won't show up in the document, and are useful for notes
% to yourself and explaining commands.
% Commenting also removes a line from the document;
% very handy for troubleshooting problems. -BTS

% As far as I know, this follows the requirements laid out in
% the 2002-2003 Senior Handbook. Ask a librarian to check the
% document before binding. -SN

%%
%% Preamble
%%
% \documentclass{<something>} must begin each LaTeX document
\documentclass[12pt,oneside]{reedthesis}
% Packages are extensions to the basic LaTeX functions. Whatever you
% want to typeset, there is probably a package out there for it.
% Chemistry (chemtex), screenplays, you name it.
% Check out CTAN to see: https://www.ctan.org/
%%
\usepackage{graphicx,latexsym}
%\usepackage{amsmath}
\usepackage{amssymb,amsthm}
\usepackage{longtable,booktabs,setspace}
\usepackage{chemarr} %% Useful for one reaction arrow, useless if you're not a chem major
\usepackage[hyphens]{url}
% Added by CII
\usepackage{hyperref}
\usepackage{lmodern}
\usepackage{float}
\floatplacement{figure}{H}
% Thanks, @Xyv
\usepackage{calc}
% End of CII addition
\usepackage{rotating}

% Next line commented out by CII
%%% \usepackage{natbib}
% Comment out the natbib line above and uncomment the following two lines to use the new
% biblatex-chicago style, for Chicago A. Also make some changes at the end where the
% bibliography is included.
%\usepackage{biblatex-chicago}
%\bibliography{thesis}

\renewcommand{\tablename}{Tabla}

% Added by CII (Thanks, Hadley!)
% Use ref for internal links
\renewcommand{\hyperref}[2][???]{\autoref{#1}}
\def\chapterautorefname{Capítulo}
\def\sectionautorefname{Sección}
\def\subsectionautorefname{Subsección}
% End of CII addition

% Added by CII
\usepackage{caption}
\captionsetup{width=5in}
% End of CII addition

% \usepackage{times} % other fonts are available like times, bookman, charter, palatino

% Syntax highlighting #22

% To pass between YAML and LaTeX the dollar signs are added by CII
\title{Caracterización morfológica, fisiológica y molecular de entradas de algodón (\emph{Gossypium hirsutum} L.) e identificación de QTL de importancia agronómica}
\author{Ing. Agr. Pablo Nahuel Dileo}
% The month and year that you submit your FINAL draft TO THE LIBRARY (May or December)
\date{Año 2025}
\division{Facultad de Ciencias Agrarias}
\advisor{Dr.~Gustavo Rubén Rodríguez}
\institution{Universidad Nacional del Nordeste}
\degree{Doctorado en Recursos Naturales}
%If you have two advisors for some reason, you can use the following
% Uncommented out by CII
\altadvisor{Dr.~Marcelo Javier Paytas}
% End of CII addition

%%% Remember to use the correct department!
\department{Doctorado}
% if you're writing a thesis in an interdisciplinary major,
% uncomment the line below and change the text as appropriate.
% check the Senior Handbook if unsure.
%\thedivisionof{The Established Interdisciplinary Committee for}
% if you want the approval page to say "Approved for the Committee",
% uncomment the next line
%\approvedforthe{Committee}

% Added by CII
%%% Copied from knitr
%% maxwidth is the original width if it's less than linewidth
%% otherwise use linewidth (to make sure the graphics do not exceed the margin)
\makeatletter
\def\maxwidth{ %
  \ifdim\Gin@nat@width>\linewidth
    \linewidth
  \else
    \Gin@nat@width
  \fi
}
\makeatother

% From {rticles}

\renewcommand{\contentsname}{Tabla de Contenidos}
% End of CII addition

\setlength{\parskip}{0pt}

% Added by CII

\providecommand{\tightlist}{%
  \setlength{\itemsep}{0pt}\setlength{\parskip}{0pt}}

\Acknowledgements{
Agradecimientos aquí..
}

\Dedication{
Dedicatoria aquí..
}

\Publications{
Lista de publicaciones a congresos y revistas aquí..
}

\Resumen{
Primer párrafo del resumen en español.

\par

Segundo párrafo del resumen aquí.
}

\Abstract{
Primer párrafo del resumen en inglés.

\par

Segundo párrafo del resumen aquí.
}

    \usepackage{graphicx}
    \usepackage{latexsym}
    \usepackage{fontspec}
    \setmainfont{Calibri}
    \usepackage{amsmath}
    \usepackage[spanish]{babel}
    \addto\captionsspanish{\renewcommand{\tablename}{Tabla}}
    \addto\captionsspanish{\renewcommand{\chaptername}{Capítulo}}
    \addto\captionsspanish{\renewcommand{\listtablename}{Índice de Tablas}}
    \addto\captionsspanish{\renewcommand{\listfigurename}{Índice de Figuras}}
    \addto\captionsspanish{\renewcommand{\contentsname}{Índice General}}
    \usepackage{setspace}
    \onehalfspacing
    \usepackage{geometry}
    \geometry{a4paper,left=3cm,right=3cm,top=2.5cm,bottom=2.5cm}
    \usepackage[style=apa, backend=biber]{biblatex}
    \addbibresource{referencias.bib}
    \usepackage{hyperref}
    \usepackage{csquotes}
    \usepackage{booktabs}
    \usepackage{longtable}
    \usepackage{array}
    \usepackage{multirow}
    \usepackage{wrapfig}
    \usepackage{float}
    \usepackage{colortbl}
    \usepackage{pdflscape}
    \usepackage{tabu}
    \usepackage{threeparttable}
    \usepackage{threeparttablex}
    \usepackage[normalem]{ulem}
    \usepackage{makecell}
    \usepackage{xcolor}
    \usepackage{fontspec}
    \usepackage{multicol}
    \usepackage{hhline}
    \newlength\Oldarrayrulewidth
    \newlength\Oldtabcolsep
    \usepackage{hyperref}
% End of CII addition
%%
%% End Preamble
%%
%
% Personalización de encabezado y pie de página 
\usepackage{fancyhdr} 
\pagestyle{fancy} 
\fancyhf{} % Limpia todos los encabezados y pies de página
\fancyhead{} % Limpia cualquier encabezado preexistente
\fancyfoot[R]{\thepage} % Coloca el número de página en la parte inferior derecha
\renewcommand{\headrulewidth}{0pt} % Eliminar línea en el encabezado 
\renewcommand{\footrulewidth}{0pt} % Eliminar línea en el pie de página


% Paquete necesario para configurar el espacio entre títulos
\usepackage{titlesec}  

% Configuración de interlineado general
\setstretch{1.5}  % Interlineado general de 1.5

% Configuración del espacio entre títulos y texto
\titlespacing*{\chapter}{0pt}{20pt}{10pt}  % Ajusta el espacio antes y después de los títulos de capítulos
\titlespacing*{\section}{0pt}{15pt}{5pt}   % Ajusta el espacio antes y después de los títulos de secciones
\titlespacing*{\subsection}{0pt}{10pt}{5pt} % Ajusta el espacio antes y después de los títulos de subsecciones

% Espacio entre párrafos
\setlength{\parskip}{1.5ex plus 0.5ex minus 0.5ex}  % Aumenta el espacio entre párrafos


\begin{document}

% Paginación en números romanos para las secciones preliminares
\frontmatter 
\pagestyle{empty} % this removes page numbers from the frontmatter
\pagenumbering{roman} 
\fancyfoot[R]{\thepage}

  \maketitle

  \begin{acknowledgements}
    Agradecimientos aquí..
    \thispagestyle{fancy} % Asegura que se utiliza el estilo de paginación fancy para esta página 
    \fancyhf{} % Limpia encabezados y pies de página 
    \fancyhead{} % Limpia cualquier encabezado preexistente 
    \fancyfoot[R]{\thepage} % Coloca el número de página en la parte inferior derecha
    %\setcounter{page}{3} % Establece el contador de página en 1
  \end{acknowledgements}

  \begin{dedication}
    Dedicatoria aquí..
    \thispagestyle{fancy} % Asegura que se utiliza el estilo de paginación fancy para esta página 
    \fancyhf{} % Limpia encabezados y pies de página 
    \fancyhead{} % Limpia cualquier encabezado preexistente 
    \fancyfoot[R]{\thepage} % Coloca el número de página en la parte inferior derecha
  \end{dedication}

  \begin{publications}
    Lista de publicaciones a congresos y revistas aquí..
    \thispagestyle{fancy} % Asegura que se utiliza el estilo de paginación fancy para esta página 
    \fancyhf{} % Limpia encabezados y pies de página 
    \fancyhead{} % Limpia cualquier encabezado preexistente 
    \fancyfoot[R]{\thepage} % Coloca el número de página en la parte inferior derecha
  \end{publications}

  \hypersetup{linkcolor=black}
  \setcounter{secnumdepth}{2}
  \setcounter{tocdepth}{2}
  \tableofcontents
  \thispagestyle{fancy} % Asegura que se utiliza el estilo de paginación fancy para esta página 
    \fancyhf{} % Limpia encabezados y pies de página 
    \fancyhead{} % Limpia cualquier encabezado preexistente 
    \fancyfoot[R]{\thepage} % Coloca el número de página en la parte inferior derecha

\chapter*{Lista de Abreviaturas}
\begin{table}[h]
    \centering
    \begin{tabular}{ll}
                \textbf{IF} & Índice de fibra \\
                \textbf{IS} & Índice de semillas \\
                \textbf{IU} & Índice de uniformidad de fibras \\
                \textbf{Mic} & Micronaire \\
                \textbf{NC} & Número de capullos \\
                \textbf{NSC} & Numero de semillas por capullo \\
                \textbf{PC} & Peso promedio de capullos \\
                \textbf{RB} & Rendimiento bruto \\
                \textbf{RF} & Rendimiento de fibra \\
                \textbf{RFD} & Rendimiento de fibra al desmote \\
                \textbf{Str} & Resistencia de las fibras \\
                \textbf{UHML} & Longitud de las fibras \\
                \textbf{zContinua...} & Otras abreviaturas \\
            \end{tabular}
\end{table}
\thispagestyle{fancy} % Asegura que se utiliza el estilo de paginación fancy para esta página 
    \fancyhf{} % Limpia encabezados y pies de página 
    \fancyhead{} % Limpia cualquier encabezado preexistente 
    \fancyfoot[R]{\thepage} % Coloca el número de página en la parte inferior derecha

  \begin{resumen}
    Primer párrafo del resumen en español.

    \par

    Segundo párrafo del resumen aquí.
    \thispagestyle{fancy} % Asegura que se utiliza el estilo de paginación fancy para esta página 
    \fancyhf{} % Limpia encabezados y pies de página 
    \fancyhead{} % Limpia cualquier encabezado preexistente 
    \fancyfoot[R]{\thepage} % Coloca el número de página en la parte inferior derecha
  \end{resumen}

  \begin{abstract}
    Primer párrafo del resumen en inglés.

    \par

    Segundo párrafo del resumen aquí.
    \thispagestyle{fancy} % Asegura que se utiliza el estilo de paginación fancy para esta página 
    \fancyhf{} % Limpia encabezados y pies de página 
    \fancyhead{} % Limpia cualquier encabezado preexistente 
    \fancyfoot[R]{\thepage} % Coloca el número de página en la parte inferior derecha
  \end{abstract}

  \listoftables
  \thispagestyle{fancy} % Asegura que se utiliza el estilo de paginación fancy para esta página 
  \fancyhf{} % Limpia encabezados y pies de página 
  \fancyhead{} % Limpia cualquier encabezado preexistente 
  \fancyfoot[R]{\thepage} % Coloca el número de página en la parte inferior derecha

  \listoffigures
  \thispagestyle{fancy} % Asegura que se utiliza el estilo de paginación fancy para esta página 
  \fancyhf{} % Limpia encabezados y pies de página 
  \fancyfoot[R]{\thepage} % Coloca el número de página en la parte inferior derecha
  \renewcommand{\headrulewidth}{0pt} % Elimina la línea en el encabezado para esta sección

% Configuración de encabezados y pies de página
\pagestyle{fancy}  % Activa el estilo fancyhdr para los encabezados y pies de página
\fancyhf{}  % Limpia los encabezados y pies de página predeterminados


% Eliminar mayúsculas en los encabezados
\renewcommand{\chaptermark}[1]{\markboth{\thechapter\ #1}{}}
\renewcommand{\sectionmark}[1]{\markright{\thesection\ #1}}

% Configuración del encabezado
\fancyhead[L]{\fontsize{10}{12}\selectfont \leftmark}  % Título del capítulo a la izquierda con tamaño 10
\fancyhead[C]{\fontsize{10}{12}\selectfont}            % (Opcional) El centro puede estar vacío con tamaño 10
\fancyhead[R]{\fontsize{10}{12}\selectfont \rightmark} % Título de la sección a la derecha con tamaño 10

% Configuración del pie de página
\fancyfoot[C]{}           % (Opcional) El centro del pie de página puede estar vacío
\fancyfoot[L]{}           % (Opcional) El pie de página izquierdo puede estar vacío
\fancyfoot[R]{\thepage}   % Número de página a la derecha en el pie de página
\fancyfoot[R]{\fontsize{10}{12}\selectfont \thepage} % Coloca el número de página en la parte inferior derecha con tamaño 10

\mainmatter % here the regular arabic numbering starts
\pagestyle{fancyplain} % turns page numbering back on
\renewcommand{\headrulewidth}{0.4pt} % Agregar línea en el encabezado


\chapter*{Introducción}\label{introducciuxf3n}
\addcontentsline{toc}{chapter}{Introducción}

El algodón (\emph{Gossypium hirsutum} L.) se cultiva en más de 80 países y desempeña un papel crucial en la producción textil como fuente de fibra. Argentina es el segundo productor de algodón de América Latina después de Brasil y desempeña un papel clave en los sistemas productivos de la región noreste del país. La producción de fibra cubre la demanda interna a la vez que facilita la exportación a países como Vietnam, Pakistán, Turquía, China, Indonesia, Colombia e India \autocite{icac2023,paytas2013}. Se han desarrollado nuevas variedades y sistemas de producción para adaptarse a diversas condiciones medioambientales. En Argentina, el Instituto Nacional de Tecnología Agropecuaria (INTA) cuenta con un programa de mejora genética del algodón que utiliza el germoplasma disponible para desarrollar variedades que mejoren el rendimiento y la calidad de la fibra. El programa ha logrado aumentar tanto el rendimiento como la calidad de la fibra y también ha incorporado la resistencia genética a enfermedades importantes como mancha angular, enfermedad azul y marchitez por fusarium, en numerosas variedades \autocite{royo2007,scarpin2022,scarpin2023}.

Recientemente, \textcite{scarpin2022} informaron de un progreso genético en el rendimiento de fibra, el porcentaje de desmote, el rendimiento de algodón bruto y el número de cápsulas (NC), mostrando una tasa media de crecimiento anual de 3,24 kg ha\textsuperscript{-1}, 0,05 \%, 4,86 kg ha\textsuperscript{-1} y 0,12 NC, respectivamente. Como resultado, las variedades más recientemente tienen un mayor rendimiento de fibra, porcentaje de desmote, rendimiento de algodón bruto y número de cápsulas que las variedades más antiguas. Estos resultados indican que el programa de mejoramiento genético del algodón en Argentina han logrado un progreso genético sustancial para el rendimiento de fibra y sus componentes. Además, \textcite{scarpin2023} enfatizaron que este progreso no resultó en una disminución de la calidad del algodón. Sin embargo, se necesita más investigación para mejorar el rendimiento de fibra, la calidad de la fibra y la adaptabilidad a diferentes ambientes. En la provincia de Santa Fe (Argentina), el cultivo de algodón se concentra en los departamentos de 9 de Julio, Vera y General Obligado, cada uno de los cuales presenta disparidades distintivas en cuanto a la composición del suelo y las condiciones ambientales. El tipo de clima en el norte de la provincia es subtropical según la clasificación climática de Köppen, con una estación seca en la región noroeste y sin estación seca en la región noreste de la provincia \autocite{anida2024}. Debido a estas diferencias, es esencial desarrollar genotipos para estas diferentes condiciones para avanzar en la mejora de los cultivos. Además, se necesitan esfuerzos adicionales para identificar genes o QTLs asociados con rasgos agronómicos y de calidad de fibra en variedades de algodón en Argentina a través de métodos de mapeo genético.

En un programa de mejora genética del algodón, combinar un alto rendimiento y calidad de la fibra es un reto clave. El primer paso crucial para un programa de mejora eficaz es caracterizar el germoplasma del algodón. \textcite{kearsey1996} destacaron la importancia de comprender las diferencias genéticas y la heredabilidad de los rasgos a la hora de cruzar plantas. La heredabilidad muestra en qué medida los rasgos están influidos por la genética y puede estimarse comparando las varianzas de las generaciones segregantes y no segregantes. En el algodón, se han notificado estimaciones de heredabilidad para varios rasgos, como el rendimiento de fibra, los componentes del rendimiento, la calidad de la fibra, la altura de la planta, el aceite de semilla, el nudo de la primera rama reproductiva, la cápsulas a prueba de tormentas, entre otros \autocite{meredith1984,tang1996,ribeiro2017,decarvalho2022,nidagundi2023}. Estos estimadores ayudan a los mejoradores a decidir la selección individual y a predecir cuánto pueden mejorar ciertos rasgos. Sin embargo, merece la pena tener en cuenta que el nivel de heredabilidad cambia en función del rasgo, la población y el ambiente en el que se cultivan.

\chapter{Caracterización morfológica}\label{rmd-basics}

\section{Introducción}\label{introducciuxf3n-1}

Aquí una breve introducción del capítulo\\

\section{Objetivo}\label{objetivo}

Caracterizar entradas de algodón del banco de germoplasma de INTA con diferente procedencia mediante caracteres morfológicos relacionados al rendimiento.

\section{Materiales y métodos}\label{materiales-y-muxe9todos}

Los ensayos se llevaron adelante en invernadero con condiciones semi-controladas de la Estación Experimental INTA Reconquista. Se utilizaron 26 entradas de \emph{Gossypium hirsutum L.}, coleccionados por el banco de germoplasma de INTA, procedentes de diversos sitios tanto nacionales como del extranjero (Tabla \ref{tab:tablaEntradas}).

\begin{table}[!h]
\centering
\caption{\label{tab:tablaEntradas}Entradas de Gossypium hirsutum L. y su procedencia}
\centering
\resizebox{\ifdim\width>\linewidth\linewidth\else\width\fi}{!}{
\begin{tabular}[t]{>{\raggedright\arraybackslash}p{8em}>{\raggedright\arraybackslash}p{12em}}
\toprule
Entradas & Procedencia\\
\midrule
BGSP-00177 & Argentina\\
BGSP-00192 & Argentina\\
BGSP-00193 & Argentina\\
BGSP-00194 & Argentina\\
BGSP-00166 & Argentina\\
\addlinespace
BGSP-00207 & Argentina\\
BGSP-00269 & Argentina\\
SP 41255 & Argentina-Línea avanzada\\
SP 6565 & Argentina-Línea avanzada\\
BGSP-00514 & Australia\\
\addlinespace
BGSP-00072 & Camerún\\
BGSP-00088 & Camerún\\
BGSP-00070 & Chad\\
BGSP-00748 & China\\
BGSP-00752 & China\\
\addlinespace
BGSP-00755 & China\\
BGSP-00759 & China\\
BGSP-00779 & China\\
BGSP-00067 & Costa de Marfil\\
BGSP-00028 & EEUU\\
\addlinespace
BGSP-00145 & EEUU\\
BGSP-00428 & EEUU\\
BGSP-00159 & EEUU\\
BGSP-00425 & EEUU\\
BGSP-00715 & Pakistán\\
\addlinespace
BGSP-00126 & Senegal\\
\bottomrule
\multicolumn{2}{l}{\textsuperscript{} aquí footnote}\\
\end{tabular}}
\end{table}

\subsection{Variables medidas:}\label{variables-medidas}

Las variables que se mencionan a continuación fueron registradas en todas las plantas \autocite{kerby2010}: i) Precocidad: se determinó tomando el porcentaje de cápsulas abiertas a 100 días después de la emergencia (DDE); ii) Altura: se midieron las plantas desde la base del tallo hasta la punta del ápice; iii) Nº de nudos: se contó el número de nudos presentes en el tallo de cada planta muestreada; iv) Nº de ramas vegetativas: se contaron las ramas vegetativas presentes en cada momento de muestreo; v) Nº de ramas reproductivas: se contaron el número de ramas reproductivas presentes en todas las plantas de cada momento de muestreo; vi) Nudo de inserción de la primera rama reproductiva: se registró el nudo donde se inserta la 1º rama reproductiva en cada planta muestreada; vii) Distancia de la primera posición al tallo principal: se midió la distancia que existe entre el tallo principal y la primera posición de la primera rama reproductiva de cada una de las plantas; viii) Área foliar: para el cálculo de esta variable, se midieron todas las hojas de las plantas a través de los equipos LICOR 3000 y LICOR 3050; ix) Dinámica de la floración: junto con las mediciones de las características morfológicas mencionadas en los puntos anteriores, se realizaron mapeos de dinámica de floración en todas las plantas seleccionadas para la partición de asimilados. Este procedimiento se realiza registrando en cada una de las posiciones reproductivas que genera la planta, la presencia de un pimpollo, una flor, una bocha verde, una cápsula abierta o un aborto \autocite{kerby1996monitoring}. Con estos datos se realizó la dinámica de floración de las diferentes variedades y porcentaje de retención final a la que llegan cada una de las plantas seleccionadas para el mapeo; x) Rendimiento: Para calcular el rendimiento de las diferentes entradas y sus componentes se efectuaron las siguientes mediciones: a) Rendimiento bruto de algodón: se recolectó la fibra-semilla de algodón de todas las cápsulas presentes en las plantas. Las muestras obtenidas fueron pesadas en balanzas de precisión y se realizó medición de la humedad de cada una de ellas; b) \% de desmote: se tomó cada muestra de algodón proveniente de las mediciones de rendimiento bruto, se realizó el desmote en una mini-desmotadora experimental y se pesó en una balanza de precisión la fibra y semillas por separado. El porcentaje de desmote fue la relación entre el peso de la fibra sobre el peso de la fibra más la semilla; c) Rendimiento de fibra: se multiplicó el rendimiento bruto de algodón por el \% de desmote obtenido; d) Nº de cápsulas por planta: se determinará dividiendo el peso total de la muestra recolectada en cada parcela con el peso por cápsula \autocite{wells1984}; e) N° semillas / cápsula: se contó el número de semillas presentes en la muestra \autocite{worley1974}; f) Fibra/semilla: este valor surge al dividir el peso de la fibra de algodón que resulta luego de desmotar los capullos por el número de semillas que tiene la muestra; g) Índice de semillas: se obtuvo al pesar 100 semillas de cada una de las muestras de las variedades de algodón \autocite{pettigrew2013}; x) Parámetros de calidad tecnológica de fibra de algodón. Para obtener estos parámetros se enviaron las muestras de fibra de algodón obtenidas al laboratorio de HVI (Uster 1000) en Reconquista, Santa Fe. Los parámetros de calidad tecnológica de fibra a evaluar fueron: Índice de Hilabilidad (SCI, por sus siglas en inglés), Micronaire (MIC), Índice de madurez (MAC), longitud promedio de la mitad superior (UHML, por sus siglas en inglés), longitud media (ML, por sus siglas en inglés), \% de uniformidad, índice de fibras cortas (SFI, por sus siglas en inglés), resistencia de fibra (Str), elongación (Elg), contenido de humedad de la muestra, color (Rd y +b), grado de color (C.G) y contenido de basura; xi) Fenología. Se registró el tiempo en días necesario para que las plantas alcancen los estados de: emergencia, 1º pimpollo, 1º flor abierta, cut out y 1º bocha abierta. El estado de cut out se determinará cuando el número de nudos por encima de la última flor blanca en el tallo sea menor que 4 \autocite{bourland1992}.

El ensayo se realizó en un diseño en bloque completamente aleatorizado, utilizando macetas de 5 litros (con una mezcla de suelo y sustrato comercial) en el cual se colocó 1 planta por maceta. Se utilizaron las entradas detalladas en las tablas N° \ref{tab:tablaEntradas}. \textbf{\texttt{Detallar\ el\ número\ de\ repeticiones}}

\subsection{Análisis estadístico}\label{anuxe1lisis-estaduxedstico}

Para la caracterización de la colección de germoplasma de algodón, se realizaron análisis de varianza (ANOVA) y análisis de correlación de Pearson de todos los rasgos medidos utilizando el software R \texttt{(R\ Core\ Team\ 2024)}. La distribución normal de cada rasgo se confirmó mediante la prueba de Shapiro-Wilk. Las diferencias estadísticas se evaluaron con un nivel de confianza del 95 \% (p ≤ 0,05) según la prueba L.S.D de Fisher, considerando las entradas como efecto fijo, mientras que el experimento y los bloques dentro de cada experimento como efectos aleatorios. A continuación, se realizó un estudio completo de la diversidad genética disponible en ese conjunto de entradas mediante el análisis de componentes principales.

\section{Resultados}\label{resultados}

\subsection{Fenología}\label{fenologuxeda}

La fenología se muestra en la tabla \ref{tab:tabla-fenologia}

\begin{table}[!h]
\centering
\caption{\label{tab:tabla-fenologia}Fenología de las entradas evaluadas}
\centering
\resizebox{\ifdim\width>\linewidth\linewidth\else\width\fi}{!}{
\begin{tabular}[t]{>{\raggedright\arraybackslash}p{8em}>{\raggedright\arraybackslash}p{12em}rrrrr}
\toprule
Entrada & Procedencia & DPP & DPF & DCO & DPCA & DFC\\
\midrule
BGSP-00159 & USA & 22 & 49 & 74 & 114 & 161\\
BGSP-00428 & USA & 23 & 49 & 74 & 113 & 163\\
BGSP-00028 & USA & 25 & 51 & 74 & 118 & 163\\
BGSP-00207 & Argentina & 23 & 52 & 79 & 135 & 164\\
BGSP-00269 & Argentina & 23 & 52 & 75 & 133 & 163\\
\addlinespace
SP 41255 & Argentina-Línea avanzada & 23 & 52 & 76 & 124 & 162\\
BGSP-00166 & Argentina & 23 & 52 & 74 & 124 & 163\\
BGSP-00145 & USA & 25 & 52 & 73 & 121 & 163\\
BGSP-00514 & Australia & 23 & 52 & 76 & 117 & 161\\
BGSP-00748 & China & 23 & 52 & 74 & 135 & NA\\
\addlinespace
BGSP-00425 & USA & 25 & 53 & 73 & 130 & 163\\
BGSP-00177 & Argentina & 26 & 53 & 75 & 127 & 164\\
BGSP-00070 & Chad & 27 & 53 & 76 & 128 & 163\\
BGSP-00755 & China & 25 & 53 & 75 & 124 & 163\\
BGSP-00126 & Senegal & 25 & 54 & 79 & 132 & 163\\
\addlinespace
BGSP-00067 & Costa de Marfil & 25 & 54 & 74 & 128 & 163\\
BGSP-00192 & Argentina & 25 & 54 & 76 & 124 & 162\\
BGSP-00759 & China & 25 & 54 & 77 & 134 & 163\\
BGSP-00072 & Camerún & 25 & 54 & 80 & 130 & 162\\
BGSP-00088 & Camerún & 25 & 54 & 78 & 135 & 163\\
\addlinespace
SP 6565 & Argentina-Línea avanzada & 25 & 55 & 79 & 131 & 162\\
BGSP-00715 & Pakistán & 24 & 55 & 78 & 134 & 163\\
BGSP-00752 & China & 25 & 55 & 79 & 135 & 162\\
BGSP-00194 & Argentina & 25 & 55 & 75 & 118 & 163\\
BGSP-00193 & Argentina & 25 & 55 & 78 & 137 & 162\\
\addlinespace
BGSP-00779 & China & 25 & 55 & 76 & 126 & 161\\
\bottomrule
\multicolumn{7}{l}{\textsuperscript{} DPP: Días a primer pimpollo, DPF: Días a primera flor, DCO: Días a Cut}\\
\multicolumn{7}{l}{Out, DPCA: Días a primer cápsula abierta, DFC: días a fin de ciclo}\\
\end{tabular}}
\end{table}

\subsection{Mapeo y descriptores morfológicos}\label{mapeo-y-descriptores-morfoluxf3gicos}

Puede incluirse resultados morfológicos de las entradas\ldots{}

\subsection{Componentes del rendimiento y calidad de fibra}\label{componentes-del-rendimiento-y-calidad-de-fibra}

\begin{table}[!h]
\centering\centering
\caption{\label{tab:tabla-rend-calidad}Valores medios, error estándar (entre paréntesis) y prueba L.S.D. (diferencia mínima significativa) de Fisher para rendimiento y sus componentes}
\centering
\resizebox{\ifdim\width>\linewidth\linewidth\else\width\fi}{!}{
\fontsize{12}{14}\selectfont
\begin{tabular}[t]{llllll}
\toprule
\multicolumn{1}{c}{ } & \multicolumn{5}{c}{Caracteres} \\
\cmidrule(l{3pt}r{3pt}){2-6}
\begingroup\fontsize{12}{14}\selectfont \textbf{Entradas}\endgroup & \begingroup\fontsize{12}{14}\selectfont \textbf{RB}\endgroup & \begingroup\fontsize{12}{14}\selectfont \textbf{RF}\endgroup & \begingroup\fontsize{12}{14}\selectfont \textbf{RFD}\endgroup & \begingroup\fontsize{12}{14}\selectfont \textbf{PC}\endgroup & \begingroup\fontsize{12}{14}\selectfont \textbf{NC}\endgroup\\
\midrule
 &  &  &  &  \vphantom{1} & \\
BGSP-00028 & 24.1 (2.0) & 8.7 (0.6) & 36.5 (0.7) & 4.3 (0.3) & 5.7 (0.3)\\
BGSP-00067 & 23.3 (3.3) & 8.2 (1.0) & 36.0 (1.9) & 2.8 (0.2) & 8.3 (1.2)\\
BGSP-00070 & 19.9 (1.3) & 7.7 (0.8) & 39.6 (0.8) & 3.4 (0.3) & 6.0 (0.6)\\
BGSP-00072 & 26.3 (2.2) & 9.8 (0.9) & 38.3 (0.6) & 3.5 (0.4) & 7.7 (0.3)\\
\addlinespace
BGSP-00088 & 31.0 (3.2) & 11.3 (1.1) & 37.1 (0.4) & 4.4 (0.2) & 7.0 (0.6)\\
BGSP-00126 & 16.0 (0.6) & 6.6 (0.3) & 42.9 (0.3) & 2.7 (0.2) & 6.0 (0.3)\\
BGSP-00145 & 16.1 (3.1) & 6.1 (1.1) & 38.6 (0.7) & 3.6 (0.6) & 4.7 (0.9)\\
BGSP-00159 & 14.3 (0.6) & 5.1 (0.2) & 36.8 (0.6) & 2.7 (0.1) & 5.2 (0.2)\\
BGSP-00166 & 18.2 (0.5) & 5.5 (0.2) & 31.1 (0.4) & 3.6 (0.1) & 5.1 (0.2)\\
\addlinespace
BGSP-00177 & 24.5 (2.2) & 7.9 (0.8) & 33.0 (0.4) & 4.2 (0.8) & 6.0 (0.6)\\
BGSP-00192 & 18.7 (3.3) & 7.1 (1.1) & 37.9 (2.0) & 3.3 (0.4) & 5.7 (0.3)\\
BGSP-00193 & 29.4 (3.9) & 10.9 (1.4) & 38.1 (1.8) & 4.2 (0.6) & 7.0 (0.0)\\
BGSP-00194 & 24.0 (3.1) & 9.6 (1.4) & 40.9 (0.8) & 3.7 (0.4) & 6.7 (0.9)\\
BGSP-00207 & 16.2 (0.6) & 6.7 (0.2) & 43.6 (0.5) & 2.8 (0.1) & 5.8 (0.2)\\
\addlinespace
BGSP-00269 & 16.9 (0.5) & 7.1 (0.3) & 43.7 (0.5) & 2.9 (0.2) & 6.1 (0.3)\\
BGSP-00425 & 15.1 (0.5) & 4.6 (0.2) & 31.5 (0.3) & 2.6 (0.1) & 5.8 (0.3)\\
BGSP-00428 & 21.4 (3.6) & 8.7 (1.4) & 40.1 (0.4) & 4.3 (0.7) & 5.0 (0.6)\\
BGSP-00514 & 28.0 (2.8) & 11.7 (1.2) & 42.3 (0.4) & 4.2 (0.2) & 6.7 (0.9)\\
BGSP-00715 & 24.0 (3.6) & 9.8 (1.2) & 41.6 (1.2) & 3.4 (0.4) & 7.0 (0.6)\\
\addlinespace
BGSP-00748 & 33.5 (1.8) & 12.3 (1.0) & 39.0 (2.4) & 4.4 (0.1) & 7.7 (0.3)\\
BGSP-00752 & 27.2 (2.1) & 11.3 (1.2) & 42.1 (1.9) & 4.1 (0.1) & 6.7 (0.3)\\
BGSP-00755 & 26.3 (2.0) & 10.0 (0.7) & 38.5 (0.8) & 4.7 (0.2) & 5.7 (0.7)\\
BGSP-00759 & 24.6 (3.6) & 8.7 (1.5) & 38.3 (1.1) & 4.7 (0.6) & 5.7 (1.5)\\
BGSP-00779 & 26.3 (1.5) & 10.4 (0.5) & 40.5 (0.7) & 3.0 (0.1) & 8.7 (0.3)\\
\addlinespace
SP 41255 & 20.7 (0.7) & 8.8 (0.4) & 44.7 (0.5) & 2.8 (0.1) & 7.3 (0.3)\\
SP 6565 & 17.3 (0.5) & 6.9 (0.2) & 41.2 (0.3) & 2.6 (0.1) & 6.8 (0.3)\\
 &  &  &  &  & \\
p-value & <0.001 & <0.001 & <0.001 & <0.001 & <0.001\\
Fisher’s L.S.D & 4.3 & 1.76 & 2.46 & 0.76 & 1.42\\
\bottomrule
\multicolumn{6}{l}{\textsuperscript{} \makecell[l]{Referencias: RB, Rendimiento bruto en g; RF, Rendimiento de fibra en \\ g; RFD, Rendimiento de fibra al desmote en \%; PC, Peso promedio de \\ capullos g; NC, Numero de capullos por planta.}}\\
\end{tabular}}
\end{table}

Las entradas de algodón mostraron diferencias significativas en todos los rasgos relacionados con el rendimiento y la calidad de la fibra (p \textless{} 0.05, Tabla \ref{tab:table-charac}).

\begin{table}[!h]
\centering\centering
\caption{\label{tab:table-charac}Mean values, standar error (in parentheses) and Fisher’s Least Significant Difference (L.S.D.) test for yield components and fibre quality.}
\centering
\resizebox{\ifdim\width>\linewidth\linewidth\else\width\fi}{!}{
\fontsize{14}{16}\selectfont
\begin{tabular}[t]{>{\raggedright\arraybackslash}p{2.5cm}>{\raggedright\arraybackslash}p{1cm}>{\raggedright\arraybackslash}p{1cm}>{\raggedright\arraybackslash}p{1cm}>{\raggedright\arraybackslash}p{1cm}>{\raggedright\arraybackslash}p{1cm}>{\raggedright\arraybackslash}p{1cm}>{\raggedright\arraybackslash}p{1cm}>{\raggedright\arraybackslash}p{1cm}>{\raggedright\arraybackslash}p{1cm}>{\raggedright\arraybackslash}p{1cm}>{\raggedright\arraybackslash}p{1cm}>{\raggedright\arraybackslash}p{1cm}}
\toprule
\multicolumn{1}{c}{ } & \multicolumn{12}{c}{Caracteres} \\
\cmidrule(l{3pt}r{3pt}){2-13}
\begingroup\fontsize{14}{16}\selectfont \textbf{Entradas}\endgroup & \begingroup\fontsize{14}{16}\selectfont \textbf{RB}\endgroup & \begingroup\fontsize{14}{16}\selectfont \textbf{RF}\endgroup & \begingroup\fontsize{14}{16}\selectfont \textbf{RFD}\endgroup & \begingroup\fontsize{14}{16}\selectfont \textbf{PC}\endgroup & \begingroup\fontsize{14}{16}\selectfont \textbf{NC}\endgroup & \begingroup\fontsize{14}{16}\selectfont \textbf{IS}\endgroup & \begingroup\fontsize{14}{16}\selectfont \textbf{IF}\endgroup & \begingroup\fontsize{14}{16}\selectfont \textbf{NSC}\endgroup & \begingroup\fontsize{14}{16}\selectfont \textbf{UHML}\endgroup & \begingroup\fontsize{14}{16}\selectfont \textbf{Str}\endgroup & \begingroup\fontsize{14}{16}\selectfont \textbf{Mic}\endgroup & \begingroup\fontsize{14}{16}\selectfont \textbf{IU}\endgroup\\
\midrule
 &  &  &  &  &  &  &  &  &  &  &  \vphantom{1} & \\
BGSP-00126 & 30.5 (6.6) & 13.2 (2.9) & 43.1 (0.4) & 3.6 (0.4) & 8.0 (1.0) & 8.7 (0.3) & 6.2 (0.3) & 25.0 (1.6) & 29.8 (0.3) & 30.4 (1.3) & 3.5 (0.4) & 83.8 (1.0)\\
BGSP-00159 & 26.2 (5.4) & 9.5 (1.9) & 36.3 (0.8) & 4.1 (0.5) & 6.2 (0.6) & 10.3 (1.0) & 5.6 (0.7) & 26.8 (0.7) & 28.4 (0.9) & 33.2 (2.1) & 3.7 (0.5) & 82.7 (0.5)\\
BGSP-00166 & 27.5 (4.3) & 8.6 (1.4) & 31.2 (0.3) & 4.6 (0.4) & 5.8 (0.5) & 11.5 (0.5) & 5.0 (0.4) & 28.9 (0.8) & 33.2 (0.5) & 37.3 (2.0) & 3.2 (0.3) & 85.8 (0.6)\\
BGSP-00207 & 30.2 (6.3) & 13.1 (2.7) & 43.5 (0.4) & 3.6 (0.4) & 8.0 (1.0) & 8.3 (0.2) & 6.2 (0.3) & 25.2 (2.0) & 28.3 (0.4) & 29.2 (0.7) & 3.4 (0.3) & 83.5 (0.4)\\
\addlinespace
BGSP-00269 & 29.8 (5.8) & 12.9 (2.5) & 43.4 (0.4) & 3.4 (0.2) & 8.6 (1.1) & 8.8 (0.4) & 6.5 (0.2) & 22.8 (1.9) & 28.3 (0.5) & 31.6 (1.2) & 4.3 (0.2) & 83.7 (0.3)\\
BGSP-00425 & 31.5 (7.4) & 11.3 (2.9) & 34.2 (1.2) & 3.6 (0.4) & 8.3 (1.1) & 7.9 (0.4) & 4.0 (0.6) & 30.6 (0.9) & 30.4 (1.0) & 31.3 (1.6) & 3.4 (0.5) & 82.2 (0.2)\\
SP-41255 & 34.2 (6.1) & 15.8 (3.0) & 45.8 (0.6) & 3.7 (0.5) & 9.2 (1.0) & 7.7 (0.1) & 6.2 (0.3) & 26.8 (2.8) & 29.8 (0.7) & 32.2 (1.2) & 3.8 (0.3) & 83.6 (0.7)\\
SP-6565 & 30.1 (5.8) & 12.5 (2.4) & 41.4 (1.0) & 3.7 (0.6) & 8.2 (0.7) & 8.9 (0.4) & 5.9 (0.2) & 25.5 (3.0) & 30.1 (0.5) & 32.5 (0.9) & 3.7 (0.3) & 84.9 (0.6)\\
 &  &  &  &  &  &  &  &  &  &  &  & \\
\addlinespace
p-value & 0.005 & <0.001 & <0.001 & 0.002 & <0.001 & <0.001 & <0.001 & 0.014 & <0.001 & 0.017 & <0.001 & 0.003\\
Fisher’s L.S.D & 3.68 & 1.91 & 1.99 & 0.57 & 1.19 & 1.35 & 0.75 & 4.04 & 1.38 & 4.03 & 0.43 & 1.66\\
\bottomrule
\multicolumn{13}{l}{\textsuperscript{} \makecell[l]{RB: Seed cotton yield in g, LY: Lint yield in g, LP: Lint percentage in \%, BW: Boll weight in g, BN: Boll number \\ per plant, SI: Seed index in g, LI: Lint index in g, SNPB: Seed number per boll, FL: Fibre length in mm, \\ FS: Fibre strength g tex\textasciicircum{}-1\textasciicircum{}, Mic: Micronaire, FU: Fibre uniformity in \%.}}\\
\end{tabular}}
\end{table}

La entrada SP-41255 presentó los valores medios más elevados para RB, RF, RFD y NC, con valores de 34.2 g planta\textsuperscript{-1}, 15.8 g planta\textsuperscript{-1}, 45.8 \%, y 9.2 capullos planta\textsuperscript{-1} respectivamente, sin embargo, presentó el valor medio más bajo de IS, con 7.7 g. BGSP-00159 tuvo la media más baja de RB con 26,2 g planta\textsuperscript{-1}. BGSP-00166 tuvo los valores medios más bajos para RF, RFD, MIC y NC con valores de 8.6 g planta\textsuperscript{-1}, 31.1 \%, 3.2 y 5.8 capullos planta\textsuperscript{-1} respectivamente, no obstante, presentó los valores medios más elevados de PC, IS, UHML, Str y IU con valores de 4.6 g capullo\textsuperscript{-1}, 11.5 g, 33.2 mm, 37.3 g tex\textsuperscript{-1} y 85.8 \% respectivamiente. BGSP-00269 presentó los valores medios más bajos de PC, NSC y UHML, con valores de 3.4 g capullo\textsuperscript{-1}, 22.8 semilla capullo\textsuperscript{-1}, y 28.3 mm respectivamente, mientras que tuvo el valor medio más alto para IF y Mic con valores de 6.5 g y 4.3. BGSP-00425 tuvo la NSC media mas alta, con un 30.6 semillas cápsula\textsuperscript{-1} mientras que los valores medios más bajos para IF y IU con valores de 4.0 g y 82.2 \%, respectivamente. BGSP-00207 presentó los valores medios más bajos de UHML y Str con valores de 28.3 mm y 29.2 g tex\textsuperscript{-1} respectivamente. En particular, las accesiones BGSP-00166 y SP-41255 fueron significativamente diferentes para casi todos los caracteres estudiados excepto para NSC (p \textless{} 0,01).

Estos resultados indican que existe variación fenotípica entre las accesiones de algodón tanto en términos de rendimiento como de calidad de la fibra.

The first principal component explained 45.61\% of the variance in the data, while the second principal component explained 21.97\% (Figure \ref{fig:img-PC}). The traits most associated with the first component were RB, Mic, RF, IF and NC, while UHML, Str, IU, IS, NSC, PC and RFD were most associated with the second component. The accessions located in the upper left quadrant were BGSP-00166, BGSP-00159 and BGSP-00425 being these accessions associated with higher values of fibre quality (UHML, Str, IU) and seed size (IS) and lower values in yield-related traits such as RB, RF, IF, NC, RFD. The accessions located in the lower right quadrant like SP-41255, BGSP-00269, BGSP-00207 and BGSP-00126, showed opposite performance to those described.

\begin{figure}
\includegraphics[width=400px]{figure/chap1/Para_compuestas/PCA_juntas} \caption{Biplot of the Principal component analysis with point representing the projections of accessions (1) and variables (2) in the space defined by the first two dimensions (Dim) or principal components.}\label{fig:img-PC}
\end{figure}

In particular, the accessions BGSP-00166 and SP-41255 presented more contrast in terms of the measured traits. These accessions showed different values for most of the measured traits related to both yield and fibre quality. Therefore, we selected those accessions to make a biparental cross and to generate a segregating population for estimating genetic parameters and then select promising phenotypes for yield and quality-related traits.

\global\setlength{\Oldarrayrulewidth}{\arrayrulewidth}

\global\setlength{\Oldtabcolsep}{\tabcolsep}

\setlength{\tabcolsep}{2pt}

\renewcommand*{\arraystretch}{1.5}



\providecommand{\ascline}[3]{\noalign{\global\arrayrulewidth #1}\arrayrulecolor[HTML]{#2}\cline{#3}}

\begin{longtable}[c]{|p{0.49in}|p{0.52in}|p{0.52in}|p{0.57in}|p{0.52in}|p{0.52in}|p{0.45in}|p{0.57in}|p{0.42in}|p{0.57in}|p{0.52in}|p{0.34in}}

\caption{\textcolor[HTML]{000000}{\fontsize{11}{13}\selectfont{\global\setmainfont{Arial}{Pearson´s}}}\textcolor[HTML]{000000}{\fontsize{11}{13}\selectfont{\global\setmainfont{Arial}{\ }}}\textcolor[HTML]{000000}{\fontsize{11}{13}\selectfont{\global\setmainfont{Arial}{correlation}}}\textcolor[HTML]{000000}{\fontsize{11}{13}\selectfont{\global\setmainfont{Arial}{\ }}}\textcolor[HTML]{000000}{\fontsize{11}{13}\selectfont{\global\setmainfont{Arial}{between}}}\textcolor[HTML]{000000}{\fontsize{11}{13}\selectfont{\global\setmainfont{Arial}{\ }}}\textcolor[HTML]{000000}{\fontsize{11}{13}\selectfont{\global\setmainfont{Arial}{the}}}\textcolor[HTML]{000000}{\fontsize{11}{13}\selectfont{\global\setmainfont{Arial}{\ }}}\textcolor[HTML]{000000}{\fontsize{11}{13}\selectfont{\global\setmainfont{Arial}{evaluated}}}\textcolor[HTML]{000000}{\fontsize{11}{13}\selectfont{\global\setmainfont{Arial}{\ }}}\textcolor[HTML]{000000}{\fontsize{11}{13}\selectfont{\global\setmainfont{Arial}{traits}}}\textcolor[HTML]{000000}{\fontsize{11}{13}\selectfont{\global\setmainfont{Arial}{\ }}}\textcolor[HTML]{000000}{\fontsize{11}{13}\selectfont{\global\setmainfont{Arial}{in}}}\textcolor[HTML]{000000}{\fontsize{11}{13}\selectfont{\global\setmainfont{Arial}{\ }}}\textcolor[HTML]{000000}{\fontsize{11}{13}\selectfont{\global\setmainfont{Arial}{the}}}\textcolor[HTML]{000000}{\fontsize{11}{13}\selectfont{\global\setmainfont{Arial}{\ }}}\textcolor[HTML]{000000}{\fontsize{11}{13}\selectfont{\global\setmainfont{Arial}{collection}}}\textcolor[HTML]{000000}{\fontsize{11}{13}\selectfont{\global\setmainfont{Arial}{\ }}}\textcolor[HTML]{000000}{\fontsize{11}{13}\selectfont{\global\setmainfont{Arial}{germplasm.}}}}\label{tab:correlation-greenhouse}\\

\ascline{1.5pt}{666666}{1-12}

\multicolumn{1}{>{\raggedright}m{\dimexpr 0.49in+0\tabcolsep}}{\textcolor[HTML]{000000}{\fontsize{11}{11}\selectfont{\global\setmainfont{Arial}{\ }}}} & \multicolumn{1}{>{\raggedright}m{\dimexpr 0.52in+0\tabcolsep}}{\textcolor[HTML]{000000}{\fontsize{11}{11}\selectfont{\global\setmainfont{Arial}{RB}}}} & \multicolumn{1}{>{\raggedright}m{\dimexpr 0.52in+0\tabcolsep}}{\textcolor[HTML]{000000}{\fontsize{11}{11}\selectfont{\global\setmainfont{Arial}{RF}}}} & \multicolumn{1}{>{\raggedright}m{\dimexpr 0.57in+0\tabcolsep}}{\textcolor[HTML]{000000}{\fontsize{11}{11}\selectfont{\global\setmainfont{Arial}{RFD}}}} & \multicolumn{1}{>{\raggedright}m{\dimexpr 0.52in+0\tabcolsep}}{\textcolor[HTML]{000000}{\fontsize{11}{11}\selectfont{\global\setmainfont{Arial}{PC}}}} & \multicolumn{1}{>{\raggedright}m{\dimexpr 0.52in+0\tabcolsep}}{\textcolor[HTML]{000000}{\fontsize{11}{11}\selectfont{\global\setmainfont{Arial}{NC}}}} & \multicolumn{1}{>{\raggedright}m{\dimexpr 0.45in+0\tabcolsep}}{\textcolor[HTML]{000000}{\fontsize{11}{11}\selectfont{\global\setmainfont{Arial}{IS}}}} & \multicolumn{1}{>{\raggedright}m{\dimexpr 0.57in+0\tabcolsep}}{\textcolor[HTML]{000000}{\fontsize{11}{11}\selectfont{\global\setmainfont{Arial}{IF}}}} & \multicolumn{1}{>{\raggedright}m{\dimexpr 0.42in+0\tabcolsep}}{\textcolor[HTML]{000000}{\fontsize{11}{11}\selectfont{\global\setmainfont{Arial}{NSC}}}} & \multicolumn{1}{>{\raggedright}m{\dimexpr 0.57in+0\tabcolsep}}{\textcolor[HTML]{000000}{\fontsize{11}{11}\selectfont{\global\setmainfont{Arial}{UHML}}}} & \multicolumn{1}{>{\raggedright}m{\dimexpr 0.52in+0\tabcolsep}}{\textcolor[HTML]{000000}{\fontsize{11}{11}\selectfont{\global\setmainfont{Arial}{Str}}}} & \multicolumn{1}{>{\raggedright}m{\dimexpr 0.34in+0\tabcolsep}}{\textcolor[HTML]{000000}{\fontsize{11}{11}\selectfont{\global\setmainfont{Arial}{Mic}}}} \\

\ascline{1.5pt}{666666}{1-12}\endfirsthead \caption[]{\textcolor[HTML]{000000}{\fontsize{11}{13}\selectfont{\global\setmainfont{Arial}{Pearson´s}}}\textcolor[HTML]{000000}{\fontsize{11}{13}\selectfont{\global\setmainfont{Arial}{\ }}}\textcolor[HTML]{000000}{\fontsize{11}{13}\selectfont{\global\setmainfont{Arial}{correlation}}}\textcolor[HTML]{000000}{\fontsize{11}{13}\selectfont{\global\setmainfont{Arial}{\ }}}\textcolor[HTML]{000000}{\fontsize{11}{13}\selectfont{\global\setmainfont{Arial}{between}}}\textcolor[HTML]{000000}{\fontsize{11}{13}\selectfont{\global\setmainfont{Arial}{\ }}}\textcolor[HTML]{000000}{\fontsize{11}{13}\selectfont{\global\setmainfont{Arial}{the}}}\textcolor[HTML]{000000}{\fontsize{11}{13}\selectfont{\global\setmainfont{Arial}{\ }}}\textcolor[HTML]{000000}{\fontsize{11}{13}\selectfont{\global\setmainfont{Arial}{evaluated}}}\textcolor[HTML]{000000}{\fontsize{11}{13}\selectfont{\global\setmainfont{Arial}{\ }}}\textcolor[HTML]{000000}{\fontsize{11}{13}\selectfont{\global\setmainfont{Arial}{traits}}}\textcolor[HTML]{000000}{\fontsize{11}{13}\selectfont{\global\setmainfont{Arial}{\ }}}\textcolor[HTML]{000000}{\fontsize{11}{13}\selectfont{\global\setmainfont{Arial}{in}}}\textcolor[HTML]{000000}{\fontsize{11}{13}\selectfont{\global\setmainfont{Arial}{\ }}}\textcolor[HTML]{000000}{\fontsize{11}{13}\selectfont{\global\setmainfont{Arial}{the}}}\textcolor[HTML]{000000}{\fontsize{11}{13}\selectfont{\global\setmainfont{Arial}{\ }}}\textcolor[HTML]{000000}{\fontsize{11}{13}\selectfont{\global\setmainfont{Arial}{collection}}}\textcolor[HTML]{000000}{\fontsize{11}{13}\selectfont{\global\setmainfont{Arial}{\ }}}\textcolor[HTML]{000000}{\fontsize{11}{13}\selectfont{\global\setmainfont{Arial}{germplasm.}}}}\label{tab:correlation-greenhouse}\\

\ascline{1.5pt}{666666}{1-12}

\multicolumn{1}{>{\raggedright}m{\dimexpr 0.49in+0\tabcolsep}}{\textcolor[HTML]{000000}{\fontsize{11}{11}\selectfont{\global\setmainfont{Arial}{\ }}}} & \multicolumn{1}{>{\raggedright}m{\dimexpr 0.52in+0\tabcolsep}}{\textcolor[HTML]{000000}{\fontsize{11}{11}\selectfont{\global\setmainfont{Arial}{RB}}}} & \multicolumn{1}{>{\raggedright}m{\dimexpr 0.52in+0\tabcolsep}}{\textcolor[HTML]{000000}{\fontsize{11}{11}\selectfont{\global\setmainfont{Arial}{RF}}}} & \multicolumn{1}{>{\raggedright}m{\dimexpr 0.57in+0\tabcolsep}}{\textcolor[HTML]{000000}{\fontsize{11}{11}\selectfont{\global\setmainfont{Arial}{RFD}}}} & \multicolumn{1}{>{\raggedright}m{\dimexpr 0.52in+0\tabcolsep}}{\textcolor[HTML]{000000}{\fontsize{11}{11}\selectfont{\global\setmainfont{Arial}{PC}}}} & \multicolumn{1}{>{\raggedright}m{\dimexpr 0.52in+0\tabcolsep}}{\textcolor[HTML]{000000}{\fontsize{11}{11}\selectfont{\global\setmainfont{Arial}{NC}}}} & \multicolumn{1}{>{\raggedright}m{\dimexpr 0.45in+0\tabcolsep}}{\textcolor[HTML]{000000}{\fontsize{11}{11}\selectfont{\global\setmainfont{Arial}{IS}}}} & \multicolumn{1}{>{\raggedright}m{\dimexpr 0.57in+0\tabcolsep}}{\textcolor[HTML]{000000}{\fontsize{11}{11}\selectfont{\global\setmainfont{Arial}{IF}}}} & \multicolumn{1}{>{\raggedright}m{\dimexpr 0.42in+0\tabcolsep}}{\textcolor[HTML]{000000}{\fontsize{11}{11}\selectfont{\global\setmainfont{Arial}{NSC}}}} & \multicolumn{1}{>{\raggedright}m{\dimexpr 0.57in+0\tabcolsep}}{\textcolor[HTML]{000000}{\fontsize{11}{11}\selectfont{\global\setmainfont{Arial}{UHML}}}} & \multicolumn{1}{>{\raggedright}m{\dimexpr 0.52in+0\tabcolsep}}{\textcolor[HTML]{000000}{\fontsize{11}{11}\selectfont{\global\setmainfont{Arial}{Str}}}} & \multicolumn{1}{>{\raggedright}m{\dimexpr 0.34in+0\tabcolsep}}{\textcolor[HTML]{000000}{\fontsize{11}{11}\selectfont{\global\setmainfont{Arial}{Mic}}}} \\

\ascline{1.5pt}{666666}{1-12}\endhead



\multicolumn{1}{>{\raggedright}m{\dimexpr 0.49in+0\tabcolsep}}{\textcolor[HTML]{000000}{\fontsize{11}{11}\selectfont{\global\setmainfont{Calibri}{RB}}}} & \multicolumn{1}{>{\raggedright}m{\dimexpr 0.52in+0\tabcolsep}}{\textcolor[HTML]{000000}{\fontsize{11}{11}\selectfont{\global\setmainfont{Calibri}{.97***}}}} & \multicolumn{1}{>{\raggedright}m{\dimexpr 0.52in+0\tabcolsep}}{\textcolor[HTML]{000000}{\fontsize{11}{11}\selectfont{\global\setmainfont{Calibri}{\ -\ }}}} & \multicolumn{1}{>{\raggedright}m{\dimexpr 0.57in+0\tabcolsep}}{\textcolor[HTML]{000000}{\fontsize{11}{11}\selectfont{\global\setmainfont{Calibri}{}}}} & \multicolumn{1}{>{\raggedright}m{\dimexpr 0.52in+0\tabcolsep}}{\textcolor[HTML]{000000}{\fontsize{11}{11}\selectfont{\global\setmainfont{Calibri}{}}}} & \multicolumn{1}{>{\raggedright}m{\dimexpr 0.52in+0\tabcolsep}}{\textcolor[HTML]{000000}{\fontsize{11}{11}\selectfont{\global\setmainfont{Calibri}{}}}} & \multicolumn{1}{>{\raggedright}m{\dimexpr 0.45in+0\tabcolsep}}{\textcolor[HTML]{000000}{\fontsize{11}{11}\selectfont{\global\setmainfont{Calibri}{}}}} & \multicolumn{1}{>{\raggedright}m{\dimexpr 0.57in+0\tabcolsep}}{\textcolor[HTML]{000000}{\fontsize{11}{11}\selectfont{\global\setmainfont{Calibri}{}}}} & \multicolumn{1}{>{\raggedright}m{\dimexpr 0.42in+0\tabcolsep}}{\textcolor[HTML]{000000}{\fontsize{11}{11}\selectfont{\global\setmainfont{Calibri}{}}}} & \multicolumn{1}{>{\raggedright}m{\dimexpr 0.57in+0\tabcolsep}}{\textcolor[HTML]{000000}{\fontsize{11}{11}\selectfont{\global\setmainfont{Calibri}{}}}} & \multicolumn{1}{>{\raggedright}m{\dimexpr 0.52in+0\tabcolsep}}{\textcolor[HTML]{000000}{\fontsize{11}{11}\selectfont{\global\setmainfont{Calibri}{}}}} & \multicolumn{1}{>{\raggedright}m{\dimexpr 0.34in+0\tabcolsep}}{\textcolor[HTML]{000000}{\fontsize{11}{11}\selectfont{\global\setmainfont{Calibri}{}}}} \\





\multicolumn{1}{>{\raggedright}m{\dimexpr 0.49in+0\tabcolsep}}{\textcolor[HTML]{000000}{\fontsize{11}{11}\selectfont{\global\setmainfont{Calibri}{RF}}}} & \multicolumn{1}{>{\raggedright}m{\dimexpr 0.52in+0\tabcolsep}}{\textcolor[HTML]{000000}{\fontsize{11}{11}\selectfont{\global\setmainfont{Calibri}{.18}}}} & \multicolumn{1}{>{\raggedright}m{\dimexpr 0.52in+0\tabcolsep}}{\textcolor[HTML]{000000}{\fontsize{11}{11}\selectfont{\global\setmainfont{Calibri}{.41**}}}} & \multicolumn{1}{>{\raggedright}m{\dimexpr 0.57in+0\tabcolsep}}{\textcolor[HTML]{000000}{\fontsize{11}{11}\selectfont{\global\setmainfont{Calibri}{\ -\ }}}} & \multicolumn{1}{>{\raggedright}m{\dimexpr 0.52in+0\tabcolsep}}{\textcolor[HTML]{000000}{\fontsize{11}{11}\selectfont{\global\setmainfont{Calibri}{}}}} & \multicolumn{1}{>{\raggedright}m{\dimexpr 0.52in+0\tabcolsep}}{\textcolor[HTML]{000000}{\fontsize{11}{11}\selectfont{\global\setmainfont{Calibri}{}}}} & \multicolumn{1}{>{\raggedright}m{\dimexpr 0.45in+0\tabcolsep}}{\textcolor[HTML]{000000}{\fontsize{11}{11}\selectfont{\global\setmainfont{Calibri}{}}}} & \multicolumn{1}{>{\raggedright}m{\dimexpr 0.57in+0\tabcolsep}}{\textcolor[HTML]{000000}{\fontsize{11}{11}\selectfont{\global\setmainfont{Calibri}{}}}} & \multicolumn{1}{>{\raggedright}m{\dimexpr 0.42in+0\tabcolsep}}{\textcolor[HTML]{000000}{\fontsize{11}{11}\selectfont{\global\setmainfont{Calibri}{}}}} & \multicolumn{1}{>{\raggedright}m{\dimexpr 0.57in+0\tabcolsep}}{\textcolor[HTML]{000000}{\fontsize{11}{11}\selectfont{\global\setmainfont{Calibri}{}}}} & \multicolumn{1}{>{\raggedright}m{\dimexpr 0.52in+0\tabcolsep}}{\textcolor[HTML]{000000}{\fontsize{11}{11}\selectfont{\global\setmainfont{Calibri}{}}}} & \multicolumn{1}{>{\raggedright}m{\dimexpr 0.34in+0\tabcolsep}}{\textcolor[HTML]{000000}{\fontsize{11}{11}\selectfont{\global\setmainfont{Calibri}{}}}} \\





\multicolumn{1}{>{\raggedright}m{\dimexpr 0.49in+0\tabcolsep}}{\textcolor[HTML]{000000}{\fontsize{11}{11}\selectfont{\global\setmainfont{Calibri}{RFD}}}} & \multicolumn{1}{>{\raggedright}m{\dimexpr 0.52in+0\tabcolsep}}{\textcolor[HTML]{000000}{\fontsize{11}{11}\selectfont{\global\setmainfont{Calibri}{.78***}}}} & \multicolumn{1}{>{\raggedright}m{\dimexpr 0.52in+0\tabcolsep}}{\textcolor[HTML]{000000}{\fontsize{11}{11}\selectfont{\global\setmainfont{Calibri}{.66***}}}} & \multicolumn{1}{>{\raggedright}m{\dimexpr 0.57in+0\tabcolsep}}{\textcolor[HTML]{000000}{\fontsize{11}{11}\selectfont{\global\setmainfont{Calibri}{-.21}}}} & \multicolumn{1}{>{\raggedright}m{\dimexpr 0.52in+0\tabcolsep}}{\textcolor[HTML]{000000}{\fontsize{11}{11}\selectfont{\global\setmainfont{Calibri}{\ -\ }}}} & \multicolumn{1}{>{\raggedright}m{\dimexpr 0.52in+0\tabcolsep}}{\textcolor[HTML]{000000}{\fontsize{11}{11}\selectfont{\global\setmainfont{Calibri}{}}}} & \multicolumn{1}{>{\raggedright}m{\dimexpr 0.45in+0\tabcolsep}}{\textcolor[HTML]{000000}{\fontsize{11}{11}\selectfont{\global\setmainfont{Calibri}{}}}} & \multicolumn{1}{>{\raggedright}m{\dimexpr 0.57in+0\tabcolsep}}{\textcolor[HTML]{000000}{\fontsize{11}{11}\selectfont{\global\setmainfont{Calibri}{}}}} & \multicolumn{1}{>{\raggedright}m{\dimexpr 0.42in+0\tabcolsep}}{\textcolor[HTML]{000000}{\fontsize{11}{11}\selectfont{\global\setmainfont{Calibri}{}}}} & \multicolumn{1}{>{\raggedright}m{\dimexpr 0.57in+0\tabcolsep}}{\textcolor[HTML]{000000}{\fontsize{11}{11}\selectfont{\global\setmainfont{Calibri}{}}}} & \multicolumn{1}{>{\raggedright}m{\dimexpr 0.52in+0\tabcolsep}}{\textcolor[HTML]{000000}{\fontsize{11}{11}\selectfont{\global\setmainfont{Calibri}{}}}} & \multicolumn{1}{>{\raggedright}m{\dimexpr 0.34in+0\tabcolsep}}{\textcolor[HTML]{000000}{\fontsize{11}{11}\selectfont{\global\setmainfont{Calibri}{}}}} \\





\multicolumn{1}{>{\raggedright}m{\dimexpr 0.49in+0\tabcolsep}}{\textcolor[HTML]{000000}{\fontsize{11}{11}\selectfont{\global\setmainfont{Calibri}{PC}}}} & \multicolumn{1}{>{\raggedright}m{\dimexpr 0.52in+0\tabcolsep}}{\textcolor[HTML]{000000}{\fontsize{11}{11}\selectfont{\global\setmainfont{Calibri}{.86***}}}} & \multicolumn{1}{>{\raggedright}m{\dimexpr 0.52in+0\tabcolsep}}{\textcolor[HTML]{000000}{\fontsize{11}{11}\selectfont{\global\setmainfont{Calibri}{.91***}}}} & \multicolumn{1}{>{\raggedright}m{\dimexpr 0.57in+0\tabcolsep}}{\textcolor[HTML]{000000}{\fontsize{11}{11}\selectfont{\global\setmainfont{Calibri}{.44**}}}} & \multicolumn{1}{>{\raggedright}m{\dimexpr 0.52in+0\tabcolsep}}{\textcolor[HTML]{000000}{\fontsize{11}{11}\selectfont{\global\setmainfont{Calibri}{.37**}}}} & \multicolumn{1}{>{\raggedright}m{\dimexpr 0.52in+0\tabcolsep}}{\textcolor[HTML]{000000}{\fontsize{11}{11}\selectfont{\global\setmainfont{Calibri}{\ -\ }}}} & \multicolumn{1}{>{\raggedright}m{\dimexpr 0.45in+0\tabcolsep}}{\textcolor[HTML]{000000}{\fontsize{11}{11}\selectfont{\global\setmainfont{Calibri}{}}}} & \multicolumn{1}{>{\raggedright}m{\dimexpr 0.57in+0\tabcolsep}}{\textcolor[HTML]{000000}{\fontsize{11}{11}\selectfont{\global\setmainfont{Calibri}{}}}} & \multicolumn{1}{>{\raggedright}m{\dimexpr 0.42in+0\tabcolsep}}{\textcolor[HTML]{000000}{\fontsize{11}{11}\selectfont{\global\setmainfont{Calibri}{}}}} & \multicolumn{1}{>{\raggedright}m{\dimexpr 0.57in+0\tabcolsep}}{\textcolor[HTML]{000000}{\fontsize{11}{11}\selectfont{\global\setmainfont{Calibri}{}}}} & \multicolumn{1}{>{\raggedright}m{\dimexpr 0.52in+0\tabcolsep}}{\textcolor[HTML]{000000}{\fontsize{11}{11}\selectfont{\global\setmainfont{Calibri}{}}}} & \multicolumn{1}{>{\raggedright}m{\dimexpr 0.34in+0\tabcolsep}}{\textcolor[HTML]{000000}{\fontsize{11}{11}\selectfont{\global\setmainfont{Calibri}{}}}} \\





\multicolumn{1}{>{\raggedright}m{\dimexpr 0.49in+0\tabcolsep}}{\textcolor[HTML]{000000}{\fontsize{11}{11}\selectfont{\global\setmainfont{Calibri}{NC}}}} & \multicolumn{1}{>{\raggedright}m{\dimexpr 0.52in+0\tabcolsep}}{\textcolor[HTML]{000000}{\fontsize{11}{11}\selectfont{\global\setmainfont{Calibri}{.09}}}} & \multicolumn{1}{>{\raggedright}m{\dimexpr 0.52in+0\tabcolsep}}{\textcolor[HTML]{000000}{\fontsize{11}{11}\selectfont{\global\setmainfont{Calibri}{-.09}}}} & \multicolumn{1}{>{\raggedright}m{\dimexpr 0.57in+0\tabcolsep}}{\textcolor[HTML]{000000}{\fontsize{11}{11}\selectfont{\global\setmainfont{Calibri}{-.51***}}}} & \multicolumn{1}{>{\raggedright}m{\dimexpr 0.52in+0\tabcolsep}}{\textcolor[HTML]{000000}{\fontsize{11}{11}\selectfont{\global\setmainfont{Calibri}{.54***}}}} & \multicolumn{1}{>{\raggedright}m{\dimexpr 0.52in+0\tabcolsep}}{\textcolor[HTML]{000000}{\fontsize{11}{11}\selectfont{\global\setmainfont{Calibri}{-.27}}}} & \multicolumn{1}{>{\raggedright}m{\dimexpr 0.45in+0\tabcolsep}}{\textcolor[HTML]{000000}{\fontsize{11}{11}\selectfont{\global\setmainfont{Calibri}{\ -\ }}}} & \multicolumn{1}{>{\raggedright}m{\dimexpr 0.57in+0\tabcolsep}}{\textcolor[HTML]{000000}{\fontsize{11}{11}\selectfont{\global\setmainfont{Calibri}{}}}} & \multicolumn{1}{>{\raggedright}m{\dimexpr 0.42in+0\tabcolsep}}{\textcolor[HTML]{000000}{\fontsize{11}{11}\selectfont{\global\setmainfont{Calibri}{}}}} & \multicolumn{1}{>{\raggedright}m{\dimexpr 0.57in+0\tabcolsep}}{\textcolor[HTML]{000000}{\fontsize{11}{11}\selectfont{\global\setmainfont{Calibri}{}}}} & \multicolumn{1}{>{\raggedright}m{\dimexpr 0.52in+0\tabcolsep}}{\textcolor[HTML]{000000}{\fontsize{11}{11}\selectfont{\global\setmainfont{Calibri}{}}}} & \multicolumn{1}{>{\raggedright}m{\dimexpr 0.34in+0\tabcolsep}}{\textcolor[HTML]{000000}{\fontsize{11}{11}\selectfont{\global\setmainfont{Calibri}{}}}} \\





\multicolumn{1}{>{\raggedright}m{\dimexpr 0.49in+0\tabcolsep}}{\textcolor[HTML]{000000}{\fontsize{11}{11}\selectfont{\global\setmainfont{Calibri}{IS}}}} & \multicolumn{1}{>{\raggedright}m{\dimexpr 0.52in+0\tabcolsep}}{\textcolor[HTML]{000000}{\fontsize{11}{11}\selectfont{\global\setmainfont{Calibri}{.55***}}}} & \multicolumn{1}{>{\raggedright}m{\dimexpr 0.52in+0\tabcolsep}}{\textcolor[HTML]{000000}{\fontsize{11}{11}\selectfont{\global\setmainfont{Calibri}{.63***}}}} & \multicolumn{1}{>{\raggedright}m{\dimexpr 0.57in+0\tabcolsep}}{\textcolor[HTML]{000000}{\fontsize{11}{11}\selectfont{\global\setmainfont{Calibri}{.62***}}}} & \multicolumn{1}{>{\raggedright}m{\dimexpr 0.52in+0\tabcolsep}}{\textcolor[HTML]{000000}{\fontsize{11}{11}\selectfont{\global\setmainfont{Calibri}{.47***}}}} & \multicolumn{1}{>{\raggedright}m{\dimexpr 0.52in+0\tabcolsep}}{\textcolor[HTML]{000000}{\fontsize{11}{11}\selectfont{\global\setmainfont{Calibri}{.47***}}}} & \multicolumn{1}{>{\raggedright}m{\dimexpr 0.45in+0\tabcolsep}}{\textcolor[HTML]{000000}{\fontsize{11}{11}\selectfont{\global\setmainfont{Calibri}{.28}}}} & \multicolumn{1}{>{\raggedright}m{\dimexpr 0.57in+0\tabcolsep}}{\textcolor[HTML]{000000}{\fontsize{11}{11}\selectfont{\global\setmainfont{Calibri}{\ -\ }}}} & \multicolumn{1}{>{\raggedright}m{\dimexpr 0.42in+0\tabcolsep}}{\textcolor[HTML]{000000}{\fontsize{11}{11}\selectfont{\global\setmainfont{Calibri}{}}}} & \multicolumn{1}{>{\raggedright}m{\dimexpr 0.57in+0\tabcolsep}}{\textcolor[HTML]{000000}{\fontsize{11}{11}\selectfont{\global\setmainfont{Calibri}{}}}} & \multicolumn{1}{>{\raggedright}m{\dimexpr 0.52in+0\tabcolsep}}{\textcolor[HTML]{000000}{\fontsize{11}{11}\selectfont{\global\setmainfont{Calibri}{}}}} & \multicolumn{1}{>{\raggedright}m{\dimexpr 0.34in+0\tabcolsep}}{\textcolor[HTML]{000000}{\fontsize{11}{11}\selectfont{\global\setmainfont{Calibri}{}}}} \\





\multicolumn{1}{>{\raggedright}m{\dimexpr 0.49in+0\tabcolsep}}{\textcolor[HTML]{000000}{\fontsize{11}{11}\selectfont{\global\setmainfont{Calibri}{IF}}}} & \multicolumn{1}{>{\raggedright}m{\dimexpr 0.52in+0\tabcolsep}}{\textcolor[HTML]{000000}{\fontsize{11}{11}\selectfont{\global\setmainfont{Calibri}{.62***}}}} & \multicolumn{1}{>{\raggedright}m{\dimexpr 0.52in+0\tabcolsep}}{\textcolor[HTML]{000000}{\fontsize{11}{11}\selectfont{\global\setmainfont{Calibri}{.54***}}}} & \multicolumn{1}{>{\raggedright}m{\dimexpr 0.57in+0\tabcolsep}}{\textcolor[HTML]{000000}{\fontsize{11}{11}\selectfont{\global\setmainfont{Calibri}{-.31*}}}} & \multicolumn{1}{>{\raggedright}m{\dimexpr 0.52in+0\tabcolsep}}{\textcolor[HTML]{000000}{\fontsize{11}{11}\selectfont{\global\setmainfont{Calibri}{.72***}}}} & \multicolumn{1}{>{\raggedright}m{\dimexpr 0.52in+0\tabcolsep}}{\textcolor[HTML]{000000}{\fontsize{11}{11}\selectfont{\global\setmainfont{Calibri}{.32*}}}} & \multicolumn{1}{>{\raggedright}m{\dimexpr 0.45in+0\tabcolsep}}{\textcolor[HTML]{000000}{\fontsize{11}{11}\selectfont{\global\setmainfont{Calibri}{.00}}}} & \multicolumn{1}{>{\raggedright}m{\dimexpr 0.57in+0\tabcolsep}}{\textcolor[HTML]{000000}{\fontsize{11}{11}\selectfont{\global\setmainfont{Calibri}{-.09}}}} & \multicolumn{1}{>{\raggedright}m{\dimexpr 0.42in+0\tabcolsep}}{\textcolor[HTML]{000000}{\fontsize{11}{11}\selectfont{\global\setmainfont{Calibri}{\ -\ }}}} & \multicolumn{1}{>{\raggedright}m{\dimexpr 0.57in+0\tabcolsep}}{\textcolor[HTML]{000000}{\fontsize{11}{11}\selectfont{\global\setmainfont{Calibri}{}}}} & \multicolumn{1}{>{\raggedright}m{\dimexpr 0.52in+0\tabcolsep}}{\textcolor[HTML]{000000}{\fontsize{11}{11}\selectfont{\global\setmainfont{Calibri}{}}}} & \multicolumn{1}{>{\raggedright}m{\dimexpr 0.34in+0\tabcolsep}}{\textcolor[HTML]{000000}{\fontsize{11}{11}\selectfont{\global\setmainfont{Calibri}{}}}} \\





\multicolumn{1}{>{\raggedright}m{\dimexpr 0.49in+0\tabcolsep}}{\textcolor[HTML]{000000}{\fontsize{11}{11}\selectfont{\global\setmainfont{Calibri}{NSC}}}} & \multicolumn{1}{>{\raggedright}m{\dimexpr 0.52in+0\tabcolsep}}{\textcolor[HTML]{000000}{\fontsize{11}{11}\selectfont{\global\setmainfont{Calibri}{-.36*}}}} & \multicolumn{1}{>{\raggedright}m{\dimexpr 0.52in+0\tabcolsep}}{\textcolor[HTML]{000000}{\fontsize{11}{11}\selectfont{\global\setmainfont{Calibri}{-.43**}}}} & \multicolumn{1}{>{\raggedright}m{\dimexpr 0.57in+0\tabcolsep}}{\textcolor[HTML]{000000}{\fontsize{11}{11}\selectfont{\global\setmainfont{Calibri}{-.49***}}}} & \multicolumn{1}{>{\raggedright}m{\dimexpr 0.52in+0\tabcolsep}}{\textcolor[HTML]{000000}{\fontsize{11}{11}\selectfont{\global\setmainfont{Calibri}{-.09}}}} & \multicolumn{1}{>{\raggedright}m{\dimexpr 0.52in+0\tabcolsep}}{\textcolor[HTML]{000000}{\fontsize{11}{11}\selectfont{\global\setmainfont{Calibri}{-.44**}}}} & \multicolumn{1}{>{\raggedright}m{\dimexpr 0.45in+0\tabcolsep}}{\textcolor[HTML]{000000}{\fontsize{11}{11}\selectfont{\global\setmainfont{Calibri}{.12}}}} & \multicolumn{1}{>{\raggedright}m{\dimexpr 0.57in+0\tabcolsep}}{\textcolor[HTML]{000000}{\fontsize{11}{11}\selectfont{\global\setmainfont{Calibri}{-.50***}}}} & \multicolumn{1}{>{\raggedright}m{\dimexpr 0.42in+0\tabcolsep}}{\textcolor[HTML]{000000}{\fontsize{11}{11}\selectfont{\global\setmainfont{Calibri}{.09}}}} & \multicolumn{1}{>{\raggedright}m{\dimexpr 0.57in+0\tabcolsep}}{\textcolor[HTML]{000000}{\fontsize{11}{11}\selectfont{\global\setmainfont{Calibri}{\ -\ }}}} & \multicolumn{1}{>{\raggedright}m{\dimexpr 0.52in+0\tabcolsep}}{\textcolor[HTML]{000000}{\fontsize{11}{11}\selectfont{\global\setmainfont{Calibri}{}}}} & \multicolumn{1}{>{\raggedright}m{\dimexpr 0.34in+0\tabcolsep}}{\textcolor[HTML]{000000}{\fontsize{11}{11}\selectfont{\global\setmainfont{Calibri}{}}}} \\





\multicolumn{1}{>{\raggedright}m{\dimexpr 0.49in+0\tabcolsep}}{\textcolor[HTML]{000000}{\fontsize{11}{11}\selectfont{\global\setmainfont{Calibri}{UHML}}}} & \multicolumn{1}{>{\raggedright}m{\dimexpr 0.52in+0\tabcolsep}}{\textcolor[HTML]{000000}{\fontsize{11}{11}\selectfont{\global\setmainfont{Calibri}{-.25}}}} & \multicolumn{1}{>{\raggedright}m{\dimexpr 0.52in+0\tabcolsep}}{\textcolor[HTML]{000000}{\fontsize{11}{11}\selectfont{\global\setmainfont{Calibri}{-.33*}}}} & \multicolumn{1}{>{\raggedright}m{\dimexpr 0.57in+0\tabcolsep}}{\textcolor[HTML]{000000}{\fontsize{11}{11}\selectfont{\global\setmainfont{Calibri}{-.43**}}}} & \multicolumn{1}{>{\raggedright}m{\dimexpr 0.52in+0\tabcolsep}}{\textcolor[HTML]{000000}{\fontsize{11}{11}\selectfont{\global\setmainfont{Calibri}{-.01}}}} & \multicolumn{1}{>{\raggedright}m{\dimexpr 0.52in+0\tabcolsep}}{\textcolor[HTML]{000000}{\fontsize{11}{11}\selectfont{\global\setmainfont{Calibri}{-.37*}}}} & \multicolumn{1}{>{\raggedright}m{\dimexpr 0.45in+0\tabcolsep}}{\textcolor[HTML]{000000}{\fontsize{11}{11}\selectfont{\global\setmainfont{Calibri}{.41**}}}} & \multicolumn{1}{>{\raggedright}m{\dimexpr 0.57in+0\tabcolsep}}{\textcolor[HTML]{000000}{\fontsize{11}{11}\selectfont{\global\setmainfont{Calibri}{-.25}}}} & \multicolumn{1}{>{\raggedright}m{\dimexpr 0.42in+0\tabcolsep}}{\textcolor[HTML]{000000}{\fontsize{11}{11}\selectfont{\global\setmainfont{Calibri}{-.08}}}} & \multicolumn{1}{>{\raggedright}m{\dimexpr 0.57in+0\tabcolsep}}{\textcolor[HTML]{000000}{\fontsize{11}{11}\selectfont{\global\setmainfont{Calibri}{.41**}}}} & \multicolumn{1}{>{\raggedright}m{\dimexpr 0.52in+0\tabcolsep}}{\textcolor[HTML]{000000}{\fontsize{11}{11}\selectfont{\global\setmainfont{Calibri}{\ -\ }}}} & \multicolumn{1}{>{\raggedright}m{\dimexpr 0.34in+0\tabcolsep}}{\textcolor[HTML]{000000}{\fontsize{11}{11}\selectfont{\global\setmainfont{Calibri}{}}}} \\





\multicolumn{1}{>{\raggedright}m{\dimexpr 0.49in+0\tabcolsep}}{\textcolor[HTML]{000000}{\fontsize{11}{11}\selectfont{\global\setmainfont{Calibri}{Str}}}} & \multicolumn{1}{>{\raggedright}m{\dimexpr 0.52in+0\tabcolsep}}{\textcolor[HTML]{000000}{\fontsize{11}{11}\selectfont{\global\setmainfont{Calibri}{.85***}}}} & \multicolumn{1}{>{\raggedright}m{\dimexpr 0.52in+0\tabcolsep}}{\textcolor[HTML]{000000}{\fontsize{11}{11}\selectfont{\global\setmainfont{Calibri}{.84***}}}} & \multicolumn{1}{>{\raggedright}m{\dimexpr 0.57in+0\tabcolsep}}{\textcolor[HTML]{000000}{\fontsize{11}{11}\selectfont{\global\setmainfont{Calibri}{.32*}}}} & \multicolumn{1}{>{\raggedright}m{\dimexpr 0.52in+0\tabcolsep}}{\textcolor[HTML]{000000}{\fontsize{11}{11}\selectfont{\global\setmainfont{Calibri}{.67***}}}} & \multicolumn{1}{>{\raggedright}m{\dimexpr 0.52in+0\tabcolsep}}{\textcolor[HTML]{000000}{\fontsize{11}{11}\selectfont{\global\setmainfont{Calibri}{.75***}}}} & \multicolumn{1}{>{\raggedright}m{\dimexpr 0.45in+0\tabcolsep}}{\textcolor[HTML]{000000}{\fontsize{11}{11}\selectfont{\global\setmainfont{Calibri}{.19}}}} & \multicolumn{1}{>{\raggedright}m{\dimexpr 0.57in+0\tabcolsep}}{\textcolor[HTML]{000000}{\fontsize{11}{11}\selectfont{\global\setmainfont{Calibri}{.73***}}}} & \multicolumn{1}{>{\raggedright}m{\dimexpr 0.42in+0\tabcolsep}}{\textcolor[HTML]{000000}{\fontsize{11}{11}\selectfont{\global\setmainfont{Calibri}{.34*}}}} & \multicolumn{1}{>{\raggedright}m{\dimexpr 0.57in+0\tabcolsep}}{\textcolor[HTML]{000000}{\fontsize{11}{11}\selectfont{\global\setmainfont{Calibri}{-.52***}}}} & \multicolumn{1}{>{\raggedright}m{\dimexpr 0.52in+0\tabcolsep}}{\textcolor[HTML]{000000}{\fontsize{11}{11}\selectfont{\global\setmainfont{Calibri}{-.30*}}}} & \multicolumn{1}{>{\raggedright}m{\dimexpr 0.34in+0\tabcolsep}}{\textcolor[HTML]{000000}{\fontsize{11}{11}\selectfont{\global\setmainfont{Calibri}{\ -\ }}}} \\





\multicolumn{1}{>{\raggedright}m{\dimexpr 0.49in+0\tabcolsep}}{\textcolor[HTML]{000000}{\fontsize{11}{11}\selectfont{\global\setmainfont{Calibri}{Mic}}}} & \multicolumn{1}{>{\raggedright}m{\dimexpr 0.52in+0\tabcolsep}}{\textcolor[HTML]{000000}{\fontsize{11}{11}\selectfont{\global\setmainfont{Calibri}{-.07}}}} & \multicolumn{1}{>{\raggedright}m{\dimexpr 0.52in+0\tabcolsep}}{\textcolor[HTML]{000000}{\fontsize{11}{11}\selectfont{\global\setmainfont{Calibri}{-.08}}}} & \multicolumn{1}{>{\raggedright}m{\dimexpr 0.57in+0\tabcolsep}}{\textcolor[HTML]{000000}{\fontsize{11}{11}\selectfont{\global\setmainfont{Calibri}{-.07}}}} & \multicolumn{1}{>{\raggedright}m{\dimexpr 0.52in+0\tabcolsep}}{\textcolor[HTML]{000000}{\fontsize{11}{11}\selectfont{\global\setmainfont{Calibri}{.09}}}} & \multicolumn{1}{>{\raggedright}m{\dimexpr 0.52in+0\tabcolsep}}{\textcolor[HTML]{000000}{\fontsize{11}{11}\selectfont{\global\setmainfont{Calibri}{-.14}}}} & \multicolumn{1}{>{\raggedright}m{\dimexpr 0.45in+0\tabcolsep}}{\textcolor[HTML]{000000}{\fontsize{11}{11}\selectfont{\global\setmainfont{Calibri}{.29*}}}} & \multicolumn{1}{>{\raggedright}m{\dimexpr 0.57in+0\tabcolsep}}{\textcolor[HTML]{000000}{\fontsize{11}{11}\selectfont{\global\setmainfont{Calibri}{.09}}}} & \multicolumn{1}{>{\raggedright}m{\dimexpr 0.42in+0\tabcolsep}}{\textcolor[HTML]{000000}{\fontsize{11}{11}\selectfont{\global\setmainfont{Calibri}{-.08}}}} & \multicolumn{1}{>{\raggedright}m{\dimexpr 0.57in+0\tabcolsep}}{\textcolor[HTML]{000000}{\fontsize{11}{11}\selectfont{\global\setmainfont{Calibri}{.40**}}}} & \multicolumn{1}{>{\raggedright}m{\dimexpr 0.52in+0\tabcolsep}}{\textcolor[HTML]{000000}{\fontsize{11}{11}\selectfont{\global\setmainfont{Calibri}{.50***}}}} & \multicolumn{1}{>{\raggedright}m{\dimexpr 0.34in+0\tabcolsep}}{\textcolor[HTML]{000000}{\fontsize{11}{11}\selectfont{\global\setmainfont{Calibri}{-.07}}}} \\

\ascline{1.5pt}{666666}{1-12}



\end{longtable}



\arrayrulecolor[HTML]{000000}

\global\setlength{\arrayrulewidth}{\Oldarrayrulewidth}

\global\setlength{\tabcolsep}{\Oldtabcolsep}

\renewcommand*{\arraystretch}{1}

Several phenotypic correlations were statistically significant (Table \ref{tab:correlation-greenhouse}). For the analysis, values greater than 0.80 were considered as strong correlation, moderate correlation were those between 0.40 to 0.80, and low were those less than 0.40. The RF showed a strong positive correlation with NC. It also showed a moderate positive correlation with PC. RFD showed a moderate positive correlation with RF and NC. However, it showed a moderate negative correlation with IS. The LF showed moderate negative correlations with RF, indicating that as the fibre length increases, the lint yield decreases.

\section{Discusión}\label{discusiuxf3n}

Aquí la discusión del capítulo 1

\section{Conclusión}\label{conclusiuxf3n}

Aquí la conclusión del capítulo 1

\chapter{Caracterización fisiológica}\label{math-sci}

\section{Introducción}\label{introducciuxf3n-2}

Aquí una breve introducción del capítulo

\section{Objetivo}\label{objetivo-1}

Evaluar procesos fisiológicos que intervienen en la determinación del rendimiento de fibra de entradas de algodón del banco de germoplasma de INTA.

\section{Materiales y métodos}\label{materiales-y-muxe9todos-1}

Variables medidas:

La medición de las variables fisiológicas se llevaron a cabo en un ensayo en invernadero. Las mediciones se realizaron en seis momentos durante el ciclo de la planta siguiendo lo propuesto por \autocite{luo2017}. Los equipos utilizados para las mediciones propuestas fueron a) Fotosíntesis: LICOR 6400; b) Contenido de clorofila: Minolta SPAD 502; c) Apertura y cierre estomático: LICOR 6400. Para las mediciones de estas variables, se utilizaron las entradas \textbf{\texttt{detallar\ entradas}} del diseño planteado en ``Caracterización morfológica''.

Por otra parte, se realizaron muestreos destructivos de plantas para la determinación de materia seca y partición de los asimilados. Los cortes se realizaron a los 30, 60, 90 y 120 días (correspondientes a los estados fenológicos 1er pimpollo, 1era flor, 1era bocha abierta y fin de ciclo). En cada uno de estos momentos se seccionaron las plantas en tallo, ramas reproductivas, ramas vegetativas, pimpollos y flores, bochas cerradas, capsulas abiertas, fibra y rebrote de tallo según el momento de muestreo. Todas las muestras de las partes de la planta fueron llevadas a estufa a 65 º C hasta peso constante (96 horas). El ensayo se realizó en un diseño en bloques al azar con arreglo en parcelas divididas, con los 4 momentos de corte y las entradas incluidas en cada parcela, con cuatro repeticiones. Se utilizaron macetas de 5 litros (con una mezcla de suelo y sustrato comercial), en el cual se colocó 1 planta por maceta.

\section{Resultados}\label{resultados-1}

Aquí los resultados del capítulo 2

\section{Discusión}\label{discusiuxf3n-1}

Aquí la discusión del capítulo 2

\section{Conclusión}\label{conclusiuxf3n-1}

Aquí la conclusión del capítulo 2

\chapter{Caracterización molecular}\label{ref-labels}

\section{Introducción}\label{introducciuxf3n-3}

Aquí una breve introducción del capítulo

\section{Objetivo}\label{objetivo-2}

Mapear QTL (Quantitative Trait Loci) asociados a caracteres de rendimiento de fibra mediante el análisis de una población segregante F\textsubscript{2} obtenida del cruzamiento de progenitores contrastantes.

\section{Materiales y métodos}\label{materiales-y-muxe9todos-2}

Se caracterizaron marcadores moleculares del tipo SSR (Single Sequence Repeats o microsatélites) asociados a QTLs de importancia agronómica. Se utilizaron 55 SSR \textbf{\texttt{chequear\ n°\ de\ marcadores}} que fueron seleccionados acorde tanto a trabajos científicos \autocite{zhang2005,shen2007,wang2007,wang2014,xia2014,wang2015,an2010,liu2012,qin2015,shi2015,su2016,zhang2016,ademe2017,liu2017,iqbal2017,li2017,baytar2018,liu2018} como la base de datos de CottonGen (\url{http://www.cottongen.org}). Dicha actividad se realizó en el laboratorio de biotecnología de INTA Reconquista. A partir de las entradas contrastantes para las características de interés, se realizaron cruzamientos y por autofecundación de las F\textsubscript{1} se obtuvo la población F\textsubscript{2}. A continuación, se detallan las etapas:

\begin{enumerate}
\def\labelenumi{\roman{enumi}.}
\item
  Obtención de la poblacion segregante para los loci de los marcadores (M) y los QTLs. Luego de la caracterización morfológica y fisiológica, las entradas fueron seleccionadas para ser utilizadas como parentales contrastantes en los respectivos cruzamientos de este estudio. Para los mismos se utilizaron las entradas de alto porcentaje de desmote como parentales masculinos. Considerando que por cada bocha se obtiene, en promedio, 20 semillas \autocite{naeem2017} se tomó una flor de cada parental para dicho cruzamiento, para la generación de la población F\textsubscript{1} (20 plantas). La población segregante (F\textsubscript{2}) resultó de la autofecundación de las F\textsubscript{1}, para llegar a una población F\textsubscript{2} aproximada de 200 plantas \autocite{bardak2018,zhang2003}.
\item
  Medición de la característica fenotípica controlada por los QTL. Se realizarán las mediciones de rendimiento y calidad de fibra detalladas en el apartado ¨Caracterización morfológica¨.
\item
  Caracterización molecular por SSR de los de los parentales y cada planta de la población F\textsubscript{2}. El ADN genómico se aisló utilizando hojas jóvenes procedentes de los progenitores y la población segregante, mediante método CTAB modificado \autocite{zhang2000,paterson1993}. Se evaluó la cantidad y calidad del ADN mediante espectrofotometría para comparación. La amplificación por PCR y la tinción con plata se realizó según \textcite{lin2005} . Los productos de PCR de los SSR se separaron en geles de poliacrilamida desnaturalizantes al \texttt{6\%} y revelados con carbonato de sodio\autocite{lin2005}.
\end{enumerate}

\section{Resultados}\label{resultados-2}

Aquí los resultados del capítulo 3

\section{Discusión}\label{discusiuxf3n-2}

Aquí la discusión del capítulo 3

\section{Conclusión}\label{conclusiuxf3n-2}

Aquí la conclusión del capítulo 3\\

\hfill\break

\chapter*{Conclusión}\label{conclusiuxf3n-3}
\addcontentsline{toc}{chapter}{Conclusión}

Las conclusiones de la tesis aquí..

\appendix

\chapter{Primer Apéndice}\label{primer-apuxe9ndice}

Sí es necesario incluir un apéndice, iría aquí..

\textbf{En capítulo \ref{rmd-basics}:}

Descripción aquí..

\chapter{Segundo Apéndice}\label{segundo-apuxe9ndice}

Este sería el segundo apéndice..

\backmatter

\chapter*{Referencias}\label{referencias}
\addcontentsline{toc}{chapter}{Referencias}

\printbibliography[heading=none]

\markboth{Referencias}{Referencias}

\noindent

\setlength{\parindent}{-0.20in}


% Index?

\end{document}
